\documentclass[a4paper, 11pt, twocolumn]{article}
\usepackage[latin1]{inputenc}
\usepackage{graphicx}
\usepackage[T1]{fontenc}
\usepackage[ngerman]{babel}
\usepackage{graphicx} 
\usepackage{layout} 
\usepackage{geometry}
\usepackage{fancyhdr}
\usepackage{expdlist}
\usepackage{makeidx}
\usepackage{tabularx}
\usepackage{multirow}
\usepackage{amsmath,amssymb,amstext}

\fancyfoot[C]{\thepage}%  Spezielle Fusszeile
\begin{document}

\title{\textbf{Tracking von Gesichtern in belebten Umgebungen}}
\author{ \textit{Kai Wolf} \\ Kai.B.Wolf@student.hs-rm.de}
\date{\today\\[5mm]
\begin{center}
\textbf{Zusammenfassung}\\[2mm]
\begin{minipage}{0.9\textwidth}
Beginn der Zusammenfassung
\end{minipage}
\end{center}
} 
\maketitle
% \vspace{15pt}

\section{Motivation}

\section{Einleitung}

\section{Tracking-Verfahren}

\subsection{Kalman Filter}

\subsection{Particle Filter}

\section{Gesichtserkennung}

\section{Evaluation}

\section{Zusammenfassung}

\bibliographystyle{plain}
\bibliography{/Users/kai/Documents/library}

\end{document}